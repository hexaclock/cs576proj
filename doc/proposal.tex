% global document settings
\documentclass[11pt, letterpaper]{article}
\usepackage[letterpaper, margin=1in]{geometry}
\usepackage[utf8]{inputenc}
\usepackage[T1]{fontenc}
\usepackage{tgtermes}
\usepackage{textcomp}
\usepackage{enumitem}
\pagestyle{empty}
\setlength{\tabcolsep}{0em}


% defined macros
\newcommand{\ProposalSection}[1]
{\noindent\textbf{#1}}


% The document actually starts here
\begin{document}
\centerline{\textbf{Proposal}}
\smallskip
\noindent\textbf{System name: }KeyLocker\\
\noindent\textbf{Members: }Hanxiong Chen, Bradford Smith, Dennis Stewart, Daniel Vinakovsky\\

\ProposalSection{Brief}

\noindent
Too many passwords to remember? Need a way to share passwords within your organization without emailing them around? KeyLocker can help you. KeyLocker is a safe for all of your keys, passwords, and other sensitive data. For your convenience, KeyLocker can also generate strong passwords on-the-fly.
\smallskip

\ProposalSection{Key Features}
\begin{itemize} \itemsep1pt \parskip0pt \parsep0pt
    \item Use AES with 256 bit encryption keys to give your passwords and other data a high level of security.
    \item Public key crypto-based authentication and encryption.
    \item Securely and effortlessly share accounts/passwords with other members of your organization.
    \item Read-only access to an offline copy of the database.
    \item Audit logging in order to establish a legal “paper” trail.
\end{itemize}

\ProposalSection{Security Elements}
\begin{itemize} \itemsep1pt \parskip0pt \parsep0pt
    \item \textbf{AES-256 encryption: }Data is encrypted before it is transferred between the client and the server.
    \item \textbf{Access control: }Users can access their personal data and modify their information.
    \item \textbf{Auto-lock \& audit report: }Each time a password is viewed, edited, or deleted the system will record the time and a log of the operation. \textit{(Bonus: The system will be locked automatically if an account tries to access the system and the authentication process fails a user-configurable number of times. That user can ask a system administrator to unlock it.)}
    \item \textbf{Password generator: }The client can generate a complex password by using a combination of numbers, letters, and symbols. The user can specify the length and what to be included in the password.
\end{itemize}

\ProposalSection{Description}

Users interact directly with the client-side application. One key function is that users can add items and view their saved passwords after they login. Each item includes a username, password, and description of the item (name of service, additional notes, etc.).

Once a request is received from the client, the server checks the account name and then uses the private key of that user to decrypt this message. The system should save items by utilizing per-user JSON files instead of a traditional SQL database. As a nice side effect, this will allow the client-side application to access a locally cached version of the JSON file associated with the user’s account and thus give the system basic offline functionality.
\end{document}
